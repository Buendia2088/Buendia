\documentclass[UTF8]{ctexart}
\usepackage{titlesec}
\usepackage{fancyhdr}
\usepackage{geometry}
\usepackage{diagbox}
\usepackage{amsmath}
\usepackage{lmodern}
\usepackage{graphicx}
\usepackage{caption,subcaption}
\usepackage{wrapfig} 
\usepackage{lineno}
\geometry{left=1.0in,right=1.0in,top=0.5in,bottom=1.0in}
\pagestyle{fancy}
\fancyhf{} 
\ctexset{
    section={
        format=\raggedright
    }
}
\begin{document}
\title{\vspace{0cm}计算机专业学习与规划}
\author{程远2234412848}
\date{}
\maketitle
\section{计算机学科各课程之间的关系图}
由于图片过大,latex显示模糊且构建极慢,故将图片放在文件夹中,图片名称为mind\_map.png。
\section{资料及链接}
1、教材网站:zlibrary(https://zh.singlelogin.re/)

2、代码高手,调试废物,数学傻子:chatGPT(https://openai.com/chatgpt/)

3、英语练习网站,代码共享网站:github(https://github.com/)

4、中国第一大学:Bilibili(https://www.bilibili.com/)

-------------------------------------以上为通用网站-------------------------------------------

5、计算机程序设计刷题:洛谷(https://www.luogu.com.cn/)

6、计算机程序设计听不懂:b站网课(https://www.bilibili.com/video/BV1et411b73Z/)

7、数据结构与算法听不懂:b站网课自己找(基本都比学校老师讲得清楚)

8、数据结构与算法往年题:康三有卖

\section{学业与职业规划}
\subsection{学业规划}


本人成绩尚好,具有一定的在内卷严重的西安交通大学良好存活的能力,也因此我决定争取保研至国内其他高校。
第一目标当然是赫赫有名的清华大学,一方面是为了其雄厚的科研能力与导师同学资源,有利于我的科研活动。另一方面也是为了一份虚名,
让我得到未来单位的认可。欲保研至清华大学,学分绩无疑是第一要务,其次是竞赛科研。欲保持学分绩班级前三甚至第一,
课程成绩、德育分、智育加分不可少。在经历高中的自律锻炼与高考的应试锻炼后,我认为课程成绩已经不是最需担心的部分。
德育分占比寥寥百分之十,况且卷德育分反倒让人更接近无德。所以我选择顺其自然,在自己喜欢与擅长的领域挣得三两分数,便是够了。
竞赛科研,是我认为需要投入更多精力的部分。参与各类竞赛,一方面是为了获奖之后加分的喜悦。另一方面,更不能忽略
其对个人能力的塑造。多年的应试教育容易僵化思维,我们不得不自寻出路调整,竞赛就是一个好手段。若是有学分绩而无能力,这显然不是一份良好的
学业规划。其次是科研,知识是我们探索世界的工具,而科研是我们利用工具进行创造的途径。所有的学习,归根结底是认识与创造。我们已认识了不少,
是时候创造了。无论是创造加分还是创造新的知识、抑或是创造将来申请的资本与面试的谈资,总归是创造,我们也总归要锻炼创造,而我倾向于在计算机视觉
或者图神经网络方面有所创造。

以上的规划服务于我保研至清华大学的宏伟学业目标,却忽略了学业的目的本身。在不思考学业目的情况下规划学业,于个人容易堕入功利与短视圈套,于国家于民族
不利于中华民族伟大复兴,这也是局部最优不一定全局最优的道理。因此,我觉得在学业规划中明确学业,或者说学习的目的是必要的。“大学之道,在明明德,在新民,在止于至善”,
这句话阐明了读书的目的。引用高中历史老师对这句话的解释,“明明德”,就是做个好人,就是分得清美丑善恶,确立自己内在的核心价值观。“新民”,这里有两层意思,
一是革新自我,促进自身的不断完善;二是推动社会进步,实现自我之于社会的价值。“止于至善”。这里的止,指的是脚指头,即“要到达的地方”,也就是目标。所以,止于至善,
可以理解为“实现理想中的完美自我”,并进一步创造完美社会。如何达到上述的境界与目标呢?有两个途径,缺一不可。第一个途径是获取与应用知识,所有的知识都是我们理解与改造世界的一种途径,
也是我们认识自我的工具。知识给我们带来的思维的愉悦,心灵的震撼,崇高的审美。知识能够帮助我们了解世界,了解自己,变化气质。第二个途径,是关心他人,关怀人类。我们拼命成为光环的意义,
不仅在于成为理想的自己,更在于照亮他人,传递能量。

罗素曾言:“有三种简单而热烈的激情支配着我的一生,那就是对知识的追求,对爱的渴望,以及对人类苦难难以遏制的同情心。”
希望我们都能通过大学学业探索理想的自我,对知识抱以热情,走进各自的光环,去照亮他人!并珍惜命运的庇护,直面命运可能的残忍,对他人的遭遇报以同情。

\subsection{职业规划}
目前我为我自己规划的理想职业是大学教授。理想的几所学校是中大、港科技或者港中文。港大也挺好的,但我的第一学历不太够,考虑到可行性还是算了。我查了港科技近几年的AP(assistant professor)学历,
phD是基本的,AP们的phD落在以下几所:清华、港中文、哥大、卡耐基梅隆等。而港大新入职的两位AP的phD都是清华。这也能说明清华phD在港校的认可度还不错。有趣的是,港科技的AP们均没有在个人主页
标记自己的本科学历。而港大则反之,两位AP都写上了自己的第一学历:一个是清华,一个是浙大。至于中大,作为广东人梦想学府,对学历的要求却不算太高。我差不到中大助理教授的信息,不知道是否学校不予公布。
但副教授们的本科学历基本在国内排名后于或者等于中大的大学,如中山大学和西北工业大学。而博士生学历基本也少见著名的计算机学府,如美四大,清华上交浙大、新国立等。但与学历和城市发展水平相符合的,
自然是教授们的薪资,这就扯到现实的方面了。港校给教授开出的薪水大约是中大的4倍左右,香港的生活成本大概是内地的2.5倍左右,内地的个人所得税差不多是香港的2倍,这些都是要综合考虑的因素。

把大学们的比来比去之余,还是要提升自己的能力。先把保研至清华的小目标达成了,之后估计就会顺很多了。



\end{document}
