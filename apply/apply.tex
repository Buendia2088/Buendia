% !TEX program = xelatex
\documentclass[11pt]{ctexart}

% ---------- 基础与链接 ----------
\usepackage[a4paper,margin=24mm]{geometry}
\usepackage[hyphens]{url}            % 允许 URL 自动断行
\usepackage[colorlinks=true,linkcolor=black,urlcolor=blue]{hyperref}
\urlstyle{same}                      % 让 \url 使用与正文一致的字体,避免中文缺字
\hypersetup{breaklinks=true}
\Urlmuskip=0mu plus 1mu              % URL 断行更灵活

% ---------- 表格与列表 ----------
\usepackage{tabularx,booktabs,array}
\usepackage{makecell}                % 多行表头
\usepackage{enumitem}
\usepackage{titlesec}
\setlist[itemize]{left=0pt,itemsep=3pt,topsep=2pt}
\setlist[enumerate]{left=0pt,itemsep=3pt,topsep=2pt}
\titleformat{\section}{\large\bfseries}{}{0pt}{}

% ---------- 便捷条目命令 ----------
\newcommand{\Project}[6]{%
\textbf{#1} \hfill \emph{#2}\par
\emph{角色:}#3 \quad \emph{技术栈:}#4\par
\emph{亮点:}#5\par
\textbf{链接:}\href{#6}{#6}\par\vspace{4pt}
}
\newcommand{\CourseRow}[3]{#1 & #2 & #3\\}
\newcommand{\Honor}[4]{\textbf{#1} — #2 \hfill #3\par #4\par\vspace{3pt}}

% == SIMPLE DOTTED TOC (no extra packages) ==
\makeatletter
\renewcommand{\contentsname}{目录}
\renewcommand{\@dotsep}{1.6}      % 值越小点越密(默认 4.5)
\renewcommand{\@pnumwidth}{2em}   % 页码占位宽度
\renewcommand{\@tocrmarg}{3em}    % 右侧边距
\renewcommand{\l@section}{\@dottedtocline{1}{0em}{2.3em}}
\renewcommand{\l@subsection}{\@dottedtocline{2}{1.5em}{3.2em}}
\renewcommand{\l@subsubsection}{\@dottedtocline{3}{3.8em}{4.1em}}
\makeatother

% 小一号目录命令(⚠️ 这里必须是 \tableofcontents)
\newcommand{\SimpleTOC}{{\small\tableofcontents}}
% == END SIMPLE DOTTED TOC ==
\begin{document}
% ---- 目录 ----
\SimpleTOC
\medskip
\newpage


% ================= 抬头 =================
% 需要在导言区(\documentclass 后)放:
% \usepackage{ctex}          % 中文支持(XeLaTeX/LuaLaTeX)
% \usepackage{hyperref}      % 链接
% \hypersetup{colorlinks=true, urlcolor=blue}

\begin{center}
  {\zihao{2}\bfseries AI赋能专业建设试点班 报名表}\par\vspace{4pt}
  {\large \bfseries 程远}
  
  2234412848
  
  计试2301
  
  13725057675

  \href{mailto:cy1656736387@gmail.com}{cy1656736387@gmail.com}\par
  
  \href{https://github.com/Buendia2088}{https://github.com/Buendia2088}
\end{center}
\vspace{2mm}\hrule\vspace{2mm}


% ================= 1. 摘要 =================
\section{摘要}
\vspace{-2pt}
\noindent 我在计算机基础与工程实现上保持\textbf{并重}:课程方面(数据结构与算法、计算机系统导论、汇编等)成绩稳定,当前均分 \textbf{93.29}/100、专业排名 \textbf{1/41};我的科研方向是\textbf{持续学习(Continual Learning)},目前已完成 CIL/TIL/DIL 基线、CPT/DAPT/LoRA-KD/Replay/EWC 等方法的系统对比与脚本化复现。\par
\noindent 希望能利用试点班的\textbf{固定工位}与\textbf{GPU 算力}支撑使我的实验能在统一环境长期稳定运行;同时我珍惜与\textbf{任课老师面对面}的交流机会,力争将阶段性成果\textbf{打磨为论文}并尝试投稿。

\section{申请动机}
\begin{itemize}[leftmargin=*,itemsep=2pt,topsep=2pt]
  \item \textbf{固定工位:}希望在试点班获得\textbf{稳定工位}(配套显示器与基本调试设备),保持持续可用的开发与训练环境,便于长时实验与随手记录。
  \item \textbf{算力支持:}充分利用试点班提供的\textbf{GPU算力}与平台,将现有 CL for LLM 实验迁移到统一环境,保证可复现与高效迭代。
  \item \textbf{面对面交流:}重视与\textbf{任课老师/授课老师}的当面沟通与即时答疑(课后、实验室走访、office hour),在高频反馈中快速调整实验与写作方向。
  \item \textbf{成果目标:}在老师的指导下,把阶段性成果\textbf{打磨为论文}并尝试投稿。
\end{itemize}

\section{AI 能力与技术栈}
\vspace{-2pt}
\begin{itemize}[leftmargin=*,itemsep=2pt,topsep=2pt]
  \item \textbf{AI 协作开发:}把 GPT 当“结对程序员”,用于环境/依赖问题定位(Conda、Docker、HF 缓存、权限)、自动生成可运行脚手架与单元测试并迭代到可复现。
  \item \textbf{AI 驱动的实验运营:}用 LLM 生成/改写训练脚本与配置,设计批量实验计划;让其\emph{解析日志→汇总指标→产出表格/曲线}(CSV→\LaTeX{} 表、Mermaid/Matplotlib 代码),统一输出 before/after、Avg Acc、BWT、PPL 的报告。
  \item \textbf{AI 强化写作与排版:}论文段落润色与结构重组、中文/英文互译;直接生成 Overleaf-ready 的 \LaTeX{} 模块(摘要、图表、算法、参考文献以及这份报名表)。
  \item \textbf{AI 沟通产出:}按对象(老师、合作者、行政)用 LLM 起草并调整语气风格,高效完成文本任务。
  \item \textbf{框架与工具:}Python(面向工程的模块化/脚本化)、PyTorch、Transformers;能够应用梯度检查点、调度与日志归档。
  \item \textbf{训练/微调:}CPT(Continual Pre-Training)、DAPT、LoRA/PEFT、LoRA-KD、EWC、Replay、梯度投影(A-GEM 思想);能在统一脚本中做 before/after 评测(PPL、Avg Acc、BWT)。
  \item \textbf{模型与任务:}GPT-2/XLM-R 等语言模型;ResNet等表征模型;具备MNIST等基准与跨域设定的实验经验。
  \item \textbf{工程与复现:}Linux、Macos与Windows基本命令;统一环境与随机种子管理,结果表与曲线自动汇总,产出可复现实验报告与 README。
\end{itemize}
\section{近期项目 / 科研经历}
\textbf{以下内容均为本人主动自驱学习与研究}
% 统一用到的安全代码样式与表格样式
\providecommand{\code}[1]{\texttt{\detokenize{#1}}}
\newcommand{\CLTableSetup}{\setlength{\tabcolsep}{4pt}\renewcommand{\arraystretch}{1.1}\footnotesize}

% ---------- 项目 1(置顶):连续学习基线 ----------
\Project{连续学习基线实验:CIL / TIL / DIL(MNIST)}{2025}
{唯一贡献者}{Python, PyTorch, HuggingFace Transformers, PEFT/LoRA}
{基于 \code{MNIST} 构造 5 个顺序任务,在 \textbf{CIL}(共享头、任务标签未知)、\textbf{TIL}(多头、任务标签已知)与 \textbf{DIL}(分布变化)三种场景统一评测 \emph{平均准确率(Avg Acc)} 与 \emph{BWT}。方法覆盖 \code{baseline}、\code{replay}、\code{ewc}、\code{kd}、\code{proj} 及其组合。}
{https://github.com/Buendia2088/ContinualLearning/tree/main/basis}

{\small
\noindent\textbf{要点结论}\;(单次代表结果):
\begin{itemize}[leftmargin=*,itemsep=2pt,topsep=2pt]
  \item CIL:\textbf{Replay + Proj} 最优(\code{replay_proj},Avg Acc 90.0\%,BWT -8.4)
  \item DIL:\textbf{Replay} 简洁有效(\code{replay},Avg Acc 95.8\%,BWT 1.1)
  \item TIL:\textbf{KD} 最稳(\code{kd},Avg Acc 98.4\%,BWT -0.4)
\end{itemize}

\noindent\textbf{结果一:Avg Acc(\%)}\\[-4pt]
\begin{center}
\CLTableSetup
\begin{tabular}{@{}lrrr@{}}
\toprule
Method & CIL & DIL & TIL \\
\midrule
\code{baseline}      & 19.1 & 88.0 & 98.4 \\
\code{ewc}           & 19.4 & 85.3 & 98.4 \\
\code{kd}            & 46.2 & 68.4 & 98.4 \\
\code{proj}          & 19.3 & 86.7 & 98.2 \\
\code{replay}        & 89.3 & 95.8 & 96.4 \\
\code{replay_kd}     & 86.6 & 90.9 & 96.7 \\
\code{replay_ewc}    & 89.6 & 95.4 & 96.2 \\
\code{replay_proj}   & 90.0 & 95.3 & 96.8 \\
\code{replay_ewc_kd} & 87.4 & 91.3 & 95.5 \\
\code{ewc_kd}        & 47.3 & 9.8  & 98.3 \\
\bottomrule
\end{tabular}
\end{center}

\noindent\textbf{结果二:BWT(百分点)}\\[-4pt]
\begin{center}
\CLTableSetup
\begin{tabular}{@{}lrrr@{}}
\toprule
Method & CIL & DIL & TIL \\
\midrule
\code{baseline}      & -99.1 & -7.9  & -0.5 \\
\code{ewc}           & -98.9 & -10.5 & -0.4 \\
\code{kd}            & -39.0 & -10.9 & -0.4 \\
\code{proj}          & -98.9 & -9.9  & -0.7 \\
\code{replay}        & -9.4  & 1.1   & -0.8 \\
\code{replay_kd}     & 34.6  & 12.2  & -0.2 \\
\code{replay_ewc}    & -8.8  & 0.4   & -0.9 \\
\code{replay_proj}   & -8.4  & 0.6   & 0.1 \\
\code{replay_ewc_kd} & 36.4  & 12.0  & -1.7 \\
\code{ewc_kd}        & -30.6 & -62.2 & -0.5 \\
\bottomrule
\end{tabular}
\end{center}
} % end small

% ---------- 项目 2:DAP ----------
\Project{DAP(Domain-Adaptive Pre-training)对比实验:LoRA / LoRA+Rewarm / LayerExp / LoRA+Replay / Full DAPT}{2025}
{唯一贡献者}{Transformers, PEFT-LoRA, Scheduler, Logging, Reproducibility}
{面向连续学习场景,将 DAP 与多种参数/训练策略结合,比较“新域适配”与“旧域遗忘”的权衡;统一输出 \code{dap_summary.json/csv} 与可视化。下表为以 PPL 指标的摘要:}
{https://github.com/Buendia2088/ContinualLearning/tree/main/clForLLM/dap}

{\small
\begin{center}
\CLTableSetup
\begin{tabular}{@{}lrrrrrr@{}}
\toprule
Variant & \makecell[r]{Old PPL\\(Before)} & \makecell[r]{Old PPL\\(After)} & \makecell[r]{$\Delta$ Old\\PPL} & \makecell[r]{New PPL\\(Before)} & \makecell[r]{New PPL\\(After)} & \makecell[r]{New PPL\\$\downarrow$} \\
\midrule
LoRA        & 3.199 & 3.658  & 0.459  & 90.266 & 3.586  & 86.679 \\
LoRA+Rewarm & 3.199 & 3.639  & 0.440  & 90.266 & 3.573  & 86.692 \\
LayerExp    & 3.199 & 3.346  & \textbf{0.147} & 90.266 & 12.918 & 77.347 \\
LoRA+Replay & 3.199 & 3.552  & 0.353  & 90.266 & 3.550  & 86.715 \\
Full DAPT   & 3.199 & 13.381 & 10.181 & 90.266 & 48.797 & 41.469 \\
\bottomrule
\end{tabular}
\end{center}

\noindent\textbf{要点}\;:LoRA 系在新域适配最强(New PPL $\approx$3.55–3.59),\textbf{LayerExp} 旧域遗忘最小($\Delta$+0.147),\textbf{Full DAPT} 对旧域破坏显著且新域收敛较差。
} % end small

% ---------- 项目 3:XLM-R CPT ----------
\Project{XLM-R CPT 连续预训练:LoRA / Full / Block Expansion}{2025}
{唯一贡献者}{XLM-R, PEFT-LoRA, FP16/Grad Checkpointing, Rewarm, Logging}
{在 XLM-R 基座上进行 \textbf{CPT}(Continual Pre-Training)对照:LoRA、Full(全量微调)与 Block Expansion。统一用 \code{metrics_before_after.json} 记录 old/new PPL;下表为摘要:}
{https://github.com/Buendia2088/ContinualLearning/tree/main/clForLLM/cpt}

{\small
\begin{center}
\CLTableSetup
\begin{tabular}{@{}lrrrrrr@{}}
\toprule
Variant & \makecell[r]{Old PPL\\(Before)} & \makecell[r]{Old PPL\\(After)} & \makecell[r]{$\Delta$ Old\\PPL} & \makecell[r]{New PPL\\(Before)} & \makecell[r]{New PPL\\(After)} & \makecell[r]{New PPL\\$\downarrow$} \\
\midrule
XLM-R LoRA     & 3.199 & 3.601  & 0.402  & 90.266 & 3.653  & 86.612 \\
XLM-R Full CPT & 3.199 & 18.416 & 15.217 & 90.266 & 40.944 & 49.322 \\
\bottomrule
\end{tabular}
\end{center}

\noindent\textbf{要点}\;:\textbf{LoRA} 在 \code{fp16+gradient_checkpointing+rewarm} 设置下稳定、适配强(New PPL 3.653;旧域 $\Delta$+0.402);\textbf{Full} 遗忘严重(旧域 18.416;新域仅 40.944),成本高且不推荐。
} % end small

% ================= 4. 课程与成绩 =================
% 需要在导言区(\documentclass 后)已加载:
% \usepackage{booktabs,tabularx,array}

% 可放在导言区:定义一个居中列类型 C{<宽度>}
\newcolumntype{C}[1]{>{\centering\arraybackslash}p{#1}}

% —— 成绩与概况 ——
\section{课程成绩}
\noindent 平均分:\textbf{93.29}/100 \quad 专业排名:\textbf{1}/41 \quad 已修学分:\textbf{101}\par
部分课程成绩如下:
\medskip
\renewcommand{\arraystretch}{1.12} % 行距微调

\begin{tabularx}{\textwidth}{@{}>{\raggedright\arraybackslash}X C{2.3cm}@{}}
\toprule
\textbf{课程} & \textbf{成绩}\\
\midrule
计算机程序设计 & 100\\
线性代数       & 100\\
游戏设计与开发 & 99\\
数据结构与算法 & 98\\
计算机系统导论 & 97\\
汇编语言       & 95\\
\bottomrule
\end{tabularx}



% ================= 5. 荣誉(含证书/竞赛) =================
\section{所获荣誉}

\begin{itemize}[leftmargin=*,itemsep=2pt,topsep=2pt,label={}]
  \item 2023--2024 学年本科生国家奖学金
  \item 2023--2024 年西安交通大学优秀学生
  \item 2024--2025 学年优衣库奖学金
  \item 2024--2025 学年西安交通大学新时代青少年先锋
\end{itemize}



% ================= 6. 申请动机(动机 / 目标 / 产出承诺) =================
% —— 申请动机(精简版)——


\end{document}
